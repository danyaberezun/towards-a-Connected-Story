\documentclass{llncs}




\usepackage{amssymb,amsmath,latexsym,enumerate,hyperref,proof,ndj}




\newcommand{\wkeval}{\Downarrow}

\newcommand{\nfeval}{\stackrel{\tiny nf}{\Downarrow}}

\newcommand{\expeval}{\stackrel{\tiny exp}{\Downarrow}}

\newcommand{\diveeval}{\stackrel{\tiny dive}{\Downarrow}}

\newcommand{\arity}[2]{\stackrel{\tt#1}{#2}}



\title{Status as of  
\today}
 
\author {Joint work by Danya Berezun and Neil D. Jones}

\institute{DIKU}

\begin{document}




%\setcounter{tocdepth}{3} \tableofcontents  \newpage

\setcounter{tocdepth}{3}

\maketitle

\section{Game semantics re-examined}

The starting point was the realisation  in 2015 that the game semantics for PCF can be thought of as a PCF interpreter. 

The bottom line, for me at least, is the unexpected fact that it is possible to build a lambda calculus interpreter that uses NONE of the traditional machinery: 
\bi
\item $\beta$-reduction; or 
\item environments binding variables to values; or 
\item ``closures'' or  ``thunks''  for implementation.
\ei
Using the game semantics framework, all these effects can be achieved (for a closed $\lambda$-expression)  by means of back pointers along a single ``traversal'', whose length is of the order of the length of the expression's head reduction sequence. 

This is visible in the original work on full abstraction for PCF by Hyland, Ong, Abramsky, McCusker {\em et al}
\cite{DBLP:journals/iandc/HylandO00,abramskyMcCusker97,DBLP:conf/tacs/AbramskyMJ94}\ldots. The potential operational consequences of the approach are  now becoming clearer, and look very promising. Further, game semantics may give a new line of attack on an old topic: {\em semantics-directed compiler generation}
\cite{DBLP:conf/cc/1980,DBLP:conf/cc/Schmidt80}.

The following follows in particular a recent paper \cite{ong2015} by  Ong on normalisation of the simply typed $\lambda$-calculus (henceforth called $\Lambda^{st}$). It is assumed that every free variable in the input is a constructor, with a given {\em arity}, i.e., number of arguments.

Let a {\em token} be any subexpression of $M$, the lambda expression being evaluated.
A {\em traversal} is a sequence of occurrences of 
tokens\footnote{A token may occur more than once, or not at all in s traversal.}. 
In addition, some of the tokens may have {\em back pointers} to earlier positions in the current traversal. 

In game semantics papers \cite{ong2015}\ldots, the denotation of a PCF expression is  a game strategy. When played, 
this game results in a traversal.

The result of normalisation as in \cite{ong2015} contains only tokens and back pointers, and no other  $\lambda$-syntax.



\subsection{A first  idea: specialise the PCF game semantics}



The traversal algorithm is defined by structural induction on the syntax of long form $\lambda$-expressions
(see \cite{ong2015} for a definition). 
A consequence: the algorithm can be {\em specialised} with respect to the sub-$\lambda$-expressions of 
$M$.\footnote{Specialisation is also known as {\em partial evaluation}.}



\subsection{Towards {\sc lll}, a low-level semantic language}



A partial evaluator, given expression $M$ in $\Lambda^{st}$ or $\Lambda^{untyped}$, will precompute all parts of the traversal algorithm for $M$ that depend only on its syntax. It thus produces as output a residual program containing no lambda-syntax, but only operations to construct the traversal. Further, recursive calls in the traversal algorithm may be unfolded.

Another way to say it: specialisation is used to {\em factor} the given traversal algorithm
$ \mathit{ta} : \Lambda \to \mathit{Traversals}$ into two stages:
$$
 \mathit{ta} = \mathit{tgen}\, ; \lsem\ \rsem^\Lambda 
 \mbox{\ where\  }  
       \mathit{tgen}:\Lambda \to \mbox{\sc lll} 
 \mbox{\ and\ }
 \lsem\ \rsem^\Lambda : \mbox{\sc lll} \to \mathit{Traversals}
$$

\section{Traversals for $\Lambda^{st}$}

\paragraph{Status:} \hfill

\be
\item We began by programming Ong's  traversal algorithm; and now have versions in both {\sc haskell} and {\sc scheme}. The {\sc haskell} version includes typing (done by Algorithm W, given user-defined types for free variables); conversion to eta-long form;  the traversal algorithm itself; and rconstruction of the esidual term.

\item The {\sc scheme} version is (at the time of writing this note) nearly in form suitable for automatic specialisation. To specialise it, we plan to use the {\sc unmix} system (developed by Sergei Romanenko).

\item We have implemented (by hand) an {\em {\sc lll}-generator}. This is a ``generating extension'', i.e., a program that, given an input $\lambda$-expression $M$, produces as output an {\sc lll}  program $p_M$  that, when run, will yield the traversals of $M$. Symbolically:
$$\lsem M\rsem = \lsem \lsem p_M\rsem \rsem
$$
The output $p_M$ is currently produced in the form of a {\sc scheme}  program. (For implementation  reasons we regard {\sc lll} as a subset of {\sc scheme}.)


\item In general the traversal of $M$ may be much larger than $M$ (by Statman's results it may be larger by a ``non-elementary'' amount!). 

A generating extension, though,  may be constructed to have size that is only linearly larger than $M$, satisfying 
$$|p_M| = O(|M|)$$

This is achieved by annotating as ``dynamic'' all calls of the traversal algorithm to itself that do not progress from one subexpression of $M$ to a proper subexpression. The effect is that these calls are never unfolded while producing the generating extension.

\ee


\paragraph{To do:} 

\be

\item Extend the approach to programs with input data.


\item Produce a generating extension automatically by partial evaluating a $\Lambda$-traverser,  using {\sc unmix}. 
(For more details on generating extensions and related topics, see \cite{jgs93}.)

\item Using {\sc unmix}, the programs produced by the generating extension are in {\sc scheme}. This has a practical advantage, that $p_M$ is directly executable (e.g., using {\sc racket}), but a disadvantage that $p_M$ in this form could be system-dependent. A next goal is to produce programs in the {\sc lll} language instead.

\ee

\section{Traversals for $\Lambda^{untyped}$}

\be

\item A traversal algorithm for {\em untyped} $\lambda$-expression $M$ has been implemented in {\sc haskell}. It is more complex than Ong's evaluator, using in all four back pointers. The net effect is that an arbitrary untyped $\lambda$-expression can be translated into {\sc lll}.

\item This traversal algorithm is also defined by structural recursion on its input $\lambda$-expression's syntax.

\item We do not yet have a {\sc scheme} version or a generating extension.

\ee


\section{Separating program from data in $\Lambda^{st}$ or $\Lambda^{untyped}$}

(This is current work.) The model is to regard a computation of $\lambda$-expression $M$ on input $d$ as a game between the {\sc lll}-codes for $M$ and $d$. We have promising examples. 

One is multiplication of Church numerals. An interesting fact: neither the multiplication program (the usual $\lambda$-calculus definition {\tt mul}) nor the data contain loops; but  {\tt mul} is compilable into an {\sc lll}-program with two nested loops that, when applied to Church numerals for  values of its two inputs,  computes their product.

For this example (and presumably many more) the computation can be done entirely without back pointers.
\vair

\fl Current work is to design a {\em communicating} version of {\sc lll}, in which such program-data games can be naturally expressed. A promising lead is to apply traditional methods for compiling {\em remote function calls} in the {\sc lll} context.

\section{Semantics-directed compiler generation}

(Just speculation at the moment.)

The idea is to specify the semantics of a subject programming language (e.g., call-by-value $\lambda$-calculus, imperative languages, etc.) by giving a mapping from source programs into {\sc lll}. A first example: express
$$
\lsem\ \rsem^\Lambda : \Lambda^{untyped} \to \mbox{\sc lll}
$$ by compositional rules with no variable environments.



\section{About the references} 

The following contain several works not mentioned in this status report, but which seem likely to become relevant as this line of research is developed further.


\bibliographystyle{plain}

\bibliography{games}

\nocite{*}

\end{document}
