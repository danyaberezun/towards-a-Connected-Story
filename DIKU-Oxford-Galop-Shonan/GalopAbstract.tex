\documentclass{article}




% SIDE MARGINS:
\oddsidemargin -10pt    %   Note that \oddsidemargin = \evensidemargin
\evensidemargin -10pt
\marginparwidth 20pt 
\marginparsep 5pt       % Horizontal space between outer margin and 
                        % marginal note
% VERTICAL SPACING:        
                         %Top of page:
\topmargin -100pt           %    Nominal distance from top of page to top of
                         %    box containing running head.
% DIMENSION OF TEXT:
\textheight 270mm        % Height of text (including footnotes and figures, 
                         % excluding running head and foot).
\textwidth 170mm         % Width of text line.

\usepackage{amssymb,amsmath,latexsym,enumerate,hyperref,proof,ndj}




\newcommand{\wkeval}{\Downarrow}

\newcommand{\nfeval}{\stackrel{\tiny nf}{\Downarrow}}

\newcommand{\expeval}{\stackrel{\tiny exp}{\Downarrow}}

\newcommand{\diveeval}{\stackrel{\tiny dive}{\Downarrow}}

\newcommand{\arity}[2]{\stackrel{\tt#1}{#2}}



\title{Partial Evaluation and Normalisation by Traversals}
 
\author {Joint work by Daniil 
Berezun\footnote{JetBrains and St. Petersburg State University (Russia)} \ and Neil D. 
Jones\footnote{DIKU, University of Copenhagen (Denmark)}}



\begin{document}




%\setcounter{tocdepth}{3} \tableofcontents  \newpage

\setcounter{tocdepth}{3}
%\setcounter{chapter}{0}

\maketitle
\vspace{-6mm}

\paragraph{Game semantics re-examined.}

A starting point: the game semantics for PCF can be thought of as a PCF interpreter. 
In game semantics papers 
\cite{
abramskyMcCusker97,
DBLP:conf/tacs/AbramskyMJ94,
DBLP:journals/iandc/AbramskyO93,
lmcs09,
blumong2008,
DBLP:journals/iandc/HylandO00,
DBLP:journals/tcs/KerNO02,
DBLP:conf/icfp/NeatherwayRO12,
ong2015}
the denotation of an expression is  a game strategy. When played, 
the game results in a 
traversal\footnote{  % FOOTNOTE
Let a {\em token} be any subexpression of $M$, the lambda expression being evaluated.
A {\em traversal} is a sequence of occurrences of  tokens. 
Some tokens  have {\em back pointers} to earlier positions in the current traversal.  
A token may occur more than once, or not at all in a traversal. 
The size of the traversals: of the order of the length of the expression's head linear reduction sequence. 
}.  % END OF FOOTNOTE
Ong's recent paper \cite{ong2015} normalises a simply typed $\lambda$-expression using   traversals.

A surprising consequence: it is possible to build a lambda calculus interpreter with {\bf none} of the traditional 
implementation machinery: 
$\beta$-reduction;   environments binding variables to values; and  ``closures'' and  ``thunks''  for  function calls and parameters.
(This was implicitly  visible  in early work on full abstraction for PCF.)

A new angle on game semantics: 
It  looks very promising to study its operational consequences. 
Further, this may give a new line of attack on an old topic: 
{\em semantics-directed compiler generation}
\cite{DBLP:conf/cc/1980,DBLP:conf/cc/Schmidt80}.




\paragraph{An  idea: specialise a traversal-based normaliser.}



Ong's  algorithm \cite{ong2015} is defined by structural recursion on the syntax of (the eta-long form of) a $\lambda$-expression $M$. 
Consequence: the  algorithm can be {\em specialised} with respect to the sub-$\lambda$-expressions of 
$M$. (Specialisation is also known as {\em partial evaluation}, see  \cite{JGS93}.)



\paragraph{An intermediate step: a low-level semantic language {\sc lll}.}



A partial evaluator, 
given a program $p$ and the {\em static} portion $s$ of its input data, will precompute the parts of $p$'s computation  that depend only on $s$, and generate residual code for all other parts of $p$. 
% given expression $M$ in $\Lambda^{st}$ or $\Lambda^{untyped}$, will precompute all parts of the traversal algorithm for $M$ that depend only on its syntax. 
In the current context: specialisation is used to {\em factor} a given traversal algorithm
$ \mathit{trav} : \Lambda \to \mathit{Traversals}$ into two stages:
$$
 \mathit{trav} = \mathit{travgen}\, ; \lsem\ \rsem^{LLL}  
 \mbox{\ where\  }  
       \mathit{travgen}:\Lambda \to \mbox{\sc lll} 
 \mbox{\ and\ }
 \lsem\ \rsem^{LLL}  : \mbox{\sc lll} \to \mathit{Traversals}
$$
The specialised traversal-builder is a residual output program in language {\sc lll}. The output program  contains  no lambda-syntax; only target code to construct the traversal. 


\paragraph{Traversals for $\Lambda^{simplytyped}$.}
\hfill


We programmed Ong's  traversal algorithm  in both {\sc haskell} and {\sc scheme}. The {\sc haskell} version includes typing (Algorithm W, given user-defined types for free variables); conversion to eta-long form;  the traversal algorithm itself; and construction of the residual $\lambda$-expression.
The {\sc scheme} version is (at the time of writing) nearly in form suitable for automatic specialisation. We will use  the system {\sc unmix}  (Sergei Romanenko).

We have implemented an {\sc lll}-generator. Given an input $\lambda$-expression $M$, the generator produces as output an {\sc lll}  program $p_M$  that, when run, will yield the traversals of $M$. Symbolically:
$\lsem M\rsem = \lsem \lsem p_M\rsem^{LLL} \rsem$. 


A well-known fact:  the traversal of $M$ may be much larger than $M$. (By Statman's results it may be larger by a ``non-elementary'' amount!). It is possible, though, to construct $p_M$ so $|p_M| = O(|M|)$, i.e., $M$'s {\sc lll} equivalent has size that is only linearly larger than $M$
 itself.

For specialisation, all calls of the traversal algorithm to itself that do not progress from one $M$ subexpression  to a proper subexpression are annotated as ``dynamic''. 
The motivation is  increased efficiency: no such recursive calls in the traversal-builder  will  be unfolded while producing the generator;
but {\em all other calls} will be unfolded. 

The current implementation regards {\sc lll} as a subset of {\sc scheme},
so the output $p_M$ is currently produced in the form of a {\sc scheme}  program. 
(Soon to be changed: replace {\sc scheme} by a strict 1. order subset of {\sc haskell}.)



\paragraph{Traversals for $\Lambda^{untyped}$.}
%\hfill


A traversal algorithm for {\em untyped} $\lambda$-expressions $M$ has been implemented in {\sc haskell}. It is more complex than Ong's evaluator, using  four different kinds of back pointers. The net effect is that an arbitrary untyped $\lambda$-expression can be translated into {\sc lll}.
A correctness proof is pending.

 As with Ong's evaluator, this  algorithm is also defined by structural recursion on its input $\lambda$-expression's syntax.
 Current work: apply partial evaluation to the  traversal algorithm for untyped $\lambda$-expressions.


\paragraph{Next steps:} (a) More on languages, partial evaluation and implementation. (b) Find a way to separate programs from data. Regard a computation of $\lambda$-expression $M$ on input $d$ as a {\em game} between the {\sc lll}-codes for $M$ and $d$. (c) Study the utility of {\sc lll} as an intermediate language for a semantics-directed compiler generator.

\newpage





\bibliographystyle{plain}

\bibliography{games}

\nocite{*}





\end{document}
